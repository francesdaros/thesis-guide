\documentclass[../main.tex]{subfiles}
\begin{document}
\chapter{Related Work}\label{cap:related-work}

\section{Front Page}

\textcolor{greenish}{
To modify the \textbf{first page} of the thesis (e.g., your name, degree programme, thesis title), you should edit the relevant fields in the file \textsf{main.tex}. The variables that can be customised include: \textsf{title}, \textsf{subtitle}, \textsf{thesisHeader}, \textsf{academicYear}, \textsf{author}, \textsf{advisor}, and \textsf{dedication}.
}

\textcolor{greenish}{
If you want to change the \textbf{UniUd watermark} or the \textbf{logo}, these elements are defined in the class file.  
In this case, you will need to modify the \textsf{.cls} file directly.
}

\section{Abstract and Lists}

\section{Citations and Bibliography}

\textcolor{greenish}{
This template uses \textsf{biblatex} to manage references.  
You can cite works in different ways depending on the context:
\begin{itemize}
    \item \textbf{Numeric (parenthetical) citation}:  
    Command \textsf{cite}. Example:~\cite{DBLP:journals/cor/RosGLMW25}
    \item \textbf{Textual citation with authors in the sentence}: Command \textsf{textcite}. 
    Example: \textcite{DBLP:journals/cor/RosGLMW25}
\end{itemize}
}

\textcolor{greenish}{
Make sure your references are stored in the \textsf{.bib} file and that all entries include accurate metadata (authors, title, venue, year, DOI).
}

\textcolor{greenish}{
The bibliography is generated automatically at the end of the document.  
It also includes the \textbf{page numbers where each work is cited}, allowing the reader to quickly locate citations within the thesis.
}

\section{Acronyms}

\textcolor{greenish}{
Acronyms used in the thesis should be defined in the file \textsf{doc/glossaries.tex}.  
The \textsf{glossaries} package automatically handles their first and subsequent appearances.
For example:
\begin{itemize}
    \item The first occurrence of \gls{cop} prints the full form.
    \item Later occurrences automatically use the short form, e.g., \gls{cop} again.
\end{itemize}
}

\section{Pre-Chapters and Post-Chapters}

\textcolor{greenish}{
The template includes optional front matter and back matter pages that you may enable depending on your degree requirements.
If you need an \textbf{authors’ declaration page}, uncomment the command
\textsf{\authorsdeclatationpage} in the \textsf{.cls} file.
Similarly, you can enable or disable other optional pages such as:
\begin{itemize}
    \item the \textbf{acronyms} page,
    \item the \textbf{symbols} page,
    \item or other preliminary sections included in the class file.
\end{itemize}
}

\end{document}